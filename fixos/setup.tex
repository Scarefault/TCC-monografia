\definecolor{blue}{RGB}{41,5,195}
\definecolor{darkblue}{RGB}{0, 0, 153}
\definecolor{white}{RGB}{255, 255, 255}
\definecolor{gray}{RGB}{242, 242, 242}
\definecolor{green}{RGB}{0, 153, 0}
\definecolor{red}{RGB}{255, 51, 0}
\definecolor{yellow}{RGB}{229, 230, 0}
\definecolor{purple}{RGB}{110, 0, 110}

\newcommand{\framework}{\textit{framework} }
\newcommand{\flexcpp}{\textsf{Flexc++} }
\newcommand{\bisoncpp}{\textsf{Bisonc++} }
\newcommand{\grails}{\textsf{Grails} }
\newcommand{\cpp}{\textsf{C++} }
\newcommand{\str}{\textit{string} }
\newcommand{\strs}{\textit{strings} }
\newcommand{\scanner}{\textit{scanner} }
\newcommand{\scanners}{\textit{scanners} }
\newcommand{\miniscanner}{\textit{mini scanner} }
\newcommand{\miniscanners}{\textit{mini scanners} }
\newcommand{\parser}{\textit{parser} }
\newcommand{\parsers}{\textit{parsers} }
\newcommand{\token}{\textit{token} }
\newcommand{\tokens}{\textit{tokens} }

\makeatletter
\hypersetup{
     	%pagebackref=true,
		pdftitle={\@title}, 
		pdfauthor={\@author},
    	pdfsubject={\imprimirpreambulo},
	    pdfcreator={LaTeX with abnTeX2},
		pdfkeywords={abnt}{latex}{abntex}{abntex2}{trabalho acadêmico}, 
		colorlinks=true,       		% false: boxed links; true: colored links
    	linkcolor=blue,          	% color of internal links
    	citecolor=blue,        		% color of links to bibliography
    	filecolor=magenta,      		% color of file links
		urlcolor=blue,
		bookmarksdepth=4
}
\makeatother
\setlength{\parindent}{1.3cm}
\setlength{\parskip}{0.2cm}  
\makeindex

% Definição de listagem utilizando listras por linha
\newcommand\realnumberstyle[1]{#1}
\makeatletter
\newcommand{\zebra}[3]{%
    {\realnumberstyle{#3}}%
    \begingroup
    \lst@basicstyle
    \ifodd\value{lstnumber}%
        \color{#1}%
    \else
        \color{#2}%
    \fi
        \rlap{
        \color@block{0.94\textwidth}{\ht\strutbox}{\dp\strutbox}%
        \hspace*{\lst@numbersep}%
        }%
    \endgroup
}
\makeatother

\lstdefinestyle{cppstyle}{
  basewidth={0.6em,0.45em},
  commentstyle=\color{green},
  keywordstyle=\color{darkblue},
  numberstyle=\zebra{gray}{white},
  stepnumber=1,
  stringstyle=\color{purple},
  basicstyle=\footnotesize,
  numbers=left,
  numbersep=5pt,
  mathescape=false,
  tabsize=4,
  keepspaces=true,
  fontadjust=true
}

\lstset{%
  language=C++,
  style=cppstyle,
%
  literate={á}{{\'a}}1 
  {à}{{\`a}}1 
  {ã}{{\~a}}1 
  {é}{{\'e}}1 
  {É}{{\'E}}1 
  {ê}{{\^e}}1 
  {õ}{{\~o}}1 
  {í}{{\'i}}1 
  {ó}{{\'o}}1 
  {ú}{{\'u}}1 
  {&}{{\&}}1 
  {ç}{{\c c}}1 
  {³}{{$^3$}}1 
  {\$}{{\$}}1 
  {Ω}{{$\Omega$}}1,
%
  breaklines=true,
  showstringspaces=false,
  basicstyle=\small\ttfamily,
  escapeinside={(*@}{@*)}
}

\lstdefinelanguage{Groovy}{
  keywords={
    abstract, default, if, private, this,
    boolean, do, implements, protected, 
    throw, break, double, import, public, 
    throws, byte, else, instanceof, return,
    transient, case, extends, int, short, 
    try, catch, final, interface, static, 
    void, char, finally, long, strictfp,
    volatile, class, float, native, super, 
    while, const, for, new, switch, continue,
    goto, package, synchronized, def, any, 
    as, in, with, null, true, false, static
  },
  keywords={[2]params, render,it, flash, redirect},
  morecomment=[l]{//},
  morecomment=[s]{/*}{*/},
  morestring=[b]",
  morestring=[d]',
  morestring=[b]""",
  morestring=[b]''',
  morestring=[b]/
}

\lstdefinestyle{groovystyle}{
  basewidth={0.6em,0.45em},
  commentstyle=\color{green},
  keywordstyle=\color{darkblue},
  numberstyle=\zebra{gray}{white},
  stepnumber=1,
  stringstyle=\color{purple},
  basicstyle=\footnotesize,
  numbers=left,
  numbersep=5pt,
  mathescape=true,
  tabsize=4,
  keepspaces=true,
  fontadjust=true
}

\lstset{%
  language=groovy,
  style=groovystyle,
%
  literate={á}{{\'a}}1 
  {à}{{\`a}}1 
  {ã}{{\~a}}1 
  {é}{{\'e}}1 
  {É}{{\'E}}1 
  {ê}{{\^e}}1 
  {õ}{{\~o}}1 
  {í}{{\'i}}1 
  {ó}{{\'o}}1 
  {ú}{{\'u}}1 
  {&}{{\&}}1 
  {ç}{{\c c}}1 
  {³}{{$^3$}}1 
  {Ω}{{$\Omega$}}1,
%
  breaklines=true,
  showstringspaces=false,
  basicstyle=\small\ttfamily
}

%% Tabelas
%------------------------------
\newcommand{\code}{\small\ttfamily}
\renewcommand\lstlistingname{Código}
\renewcommand\lstlistlistingname{Código}
\def\lstlistingautorefname{Código}
\renewcommand{\appendixautorefname}{Apêndice} % correção do bug do abntex2
\renewcommand{\tableautorefname}{Tabela} % correção do bug do abntex2
\renewcommand{\figureautorefname}{Figura} % correção do bug do abntex2
\newcommand{\refanexo}[1]{\hyperref[#1]{Anexo~\ref{#1}}} % correção do bug do abntex2
\newlistof{lstlistoflistings}{lol}{\lstlistlistingname}
%-------------------------------