\chapter[Metodologia]{Metodologia}

Esse capítulo aborda diversos tipos de metodologias e procedimentos de pesquisa, incluindo a que será utilizada durante o desenvolvimento da pesquisa deste trabalho. Ele está organizado em seções. Na seção 5.1 é explanado sobre as classificações de pesquisa quanto a abordagem, objetivo e procedimentos técnicos. Em seguida, na seção 5.2, é relatado o planejamento da metodologia aplicada tanto para pesquisa quanto para o desenvolvimento do software, bem como o fluxo de atividades e cronogramas.

 \section{Classificações de Metodologias de Pesquisa}
  
Pesquisar pode ser entendido como uma realização concreta de uma investigação planejada, desenvolvida e redigida de acordo com as normas da metodologia consagradas pela ciência, ou, de forma mais simples, como uma atividade para solucionar problemas \cite{kauark_2010} ( Ruiz). Considerando a forma de abordagem do problema, as pesquisas são categorizadas em pesquisa quantitativa e pesquisa qualitativa. A primeira refere-se a pesquisas que podem ser quantificáveis, ou seja, é possível ser traduzida em números e requer uso de técnicas estatísticas para sua análise. A pesquisa qualitativa é descritiva, sendo assim, não pode ser analisada de forma mensurável e sim indutivamente \cite{kauark_2010}.

 \par
  \indent Utilizando objetivo geral como critério de classificação, pode-se categorizar pesquisas em três grupos: exploratória, descritiva e explicativa. A pesquisa exploratória busca obter maior conhecimento sobre um assunto específico ainda pouco conhecido tornando-o mais explícito e orientando a formulação de hipóteses \cite{gil_2002}. Geralmente é realizada em forma de estudo de caso ou pesquisa bibliográfica (wiliam) e, para sua execução, costuma-se fazer uso de entrevistas com pessoas experientes no assunto, levantamento bibliográfico e análise de exemplos relacionados com a temática \cite{gil_2002}. 

 \par
  \indent A pesquisa descritiva propõe ao inverstigador descrever as características de uma população ou fenômeno, ou determinar a existência de relações de dependência entre variáveis \cite{gil_2002}.  Em geral, é realizada com auxílio de coleta de dados por meio de questionários e observações sistemáticas \cite{tafner_2007}. Para uma correta interpretação da pesquisa, é importante que o investigador observe, registre, analise e classifique os dados de forma que não haja interferências com os fatos observados (will). 

 \par
  \indent O terceiro grupo refere-se a pesquisa explicativa. Essa tem como objetivo explanar sobre os fatores que determinam ou contribuem com  a ocorrência de um fenômeno específico. Essa categoria de pesquisa visa compreender os motivos de um fenômeno, enquanto a exploratória e descritiva motivam-se a entender o fenômeno em si e detalha-lo \cite{gil_2002}. Comumente é realizada por meio de pesquisa \textit{ ex-post facto} e pesquisa experimental \cite{tafner_2007}. 

 \par
 \indent Embora a classificação quanto ao objetivo proporcione uma visão conceitual da pesquisa, existe a necessidade de obter uma visão a nível operativo. Nesse sentido, pode-se categorizar pesquisas quanto ao procedimento técnico utilizado que são:
 
\begin{description}
\item[Pesquisa Bibliográfica:] é realizada a partir do levantamento de referências teóricas sobre o tema. De forma geral, toda pesquisa inicia-se como uma pesquisa bibliográfica, mas há aquelas que dependem exclusivamente desse tipo de pesquisa. Em essência, a conclusão desse tipo de pesquisa é uma compilação dads publicações referentes ao tema. 

\item[Pesquisa Documental:] semelhante a pesquisa bibliográfica, mas tem como base documentos sem tratamento analítico, como tabelas, cartas fotos, pinturas, dentro outros. 

\item[Pesquisa Experimental:] seleciona grupos de assuntos coincidentes, submete-os a tratamentos diferentes, verificando as variáveis estranhas e checando se as diferenças observadas nas respostas são estatisticamente significantes.

\item[Pesquisa \textit{ex-post facto}:] investiga possíveis relações de causa e efeito entre um determinado fato identificado pelo pesquisador e um fenômeno que ocorre posteriormente. A principal característica da pesquisa \textit{ex-post facto} é o fato de os dados serem coletados após a ocorrência dos eventos.

\item[Levantamento:] utilizada em estudos exploratórios e descritivos, o levantamento pode ser de dois tipos: levantamento de uma amostra ou levantamento de uma população (também designado de censo). A coleta de dados realiza-se em ambos os casos por meio de questionários ou entrevistas.

\item[Estudo de coorte:] diz respeito a um tipo de pesquisa em que seleciona um grupo de pessoas com uma característica comum. A partir de então, o grupo é acompanhado a fim de observar e analisar o que ocorre com elas num determinado tempo.Esse tipo de estudo pode ser categorizado em dois tipos: estudo retrospectivo e estudo prospectivo. O primeiro ocorre com a análise de dados históricos, enquanto o segundo refere-se a uma análise com dados atuais.   

\item[Estudo de caso:] pode ser caracterizado de acordo com o estudo de uma entidade bem definida como um programa, uma instituição, um sistema educativo, uma pessoa ou uma unidade social. Visa conhecer em profundidade o seu "como" e os seus "porquês", evidenciando a sua unidade e identidade própria. 

\item[Estudo de campo:] similar ao levantamento, entretanto o estudo de campo busca maior aprofundamento nas questões propostas. O pesquisador deve ser imerso na comunidade a ser estudada de forma a obter maior conhecimento sobre as regras que regem o grupo. Esse tipo de estudo pode ser realizado com observação direta das atividades, entrevistas e análise de documentos, fotografias e filmagens \cite{tafner_2007}. 

\item[Pesquisa-ação:] esse tipo de pesquisa pressupõe uma participação planejada do pesquisador na situação problemática a ser investigada. Recorre a uma metodologia sistemática, no sentido de transformar as realidades observadas, a partir da sua compreensão, conhecimento e compromisso para a ação dos elementos envolvidos na pesquisa. O processo de pesquisa-ação envolve o planejamento, diagnóstico, a ação, a observação e a reflexão, num ciclo permanente.

\item[Pesquisa participante:] similar a pesquisa-ação diferindo quanto ao planejamento da ação. Enquanto a pesquisa-ação requer um planejamento anterior a ação do pesquisador, a pesquisa participante não pressupõe essa atividade. Com a finalidade de adquirir maior compreensão do grupo, o investigador participa da comunidade e suas atividades. A finalidade do estudo e a identidade do investigador deve ser de ciência para todos os envolvidos \cite{tafner_2007}.
\end{description}

 \section{Planejamento da Metodologia Aplicada}
  
Utilizando o estudo descrito anteriormente como base, foi definido a metodologia de pesquisa que será utilizada durante o desenvolvimento do trabalho de conclusão de curso.  Foi definida também uma adaptação de metodologia para o desenvolvimento do \textit{framework}, embasada em práticas do Scrum e do XP.  
  
  \subsection{Metodologia de Pesquisa} \label{Metodologia de Pesquisa}
  Tendo em vista o objetivo geral do trabalho, o desenvolvimento de um \textit{framework} para a geração de testes unitários, observa-se que o modelo de pesquisa que mais se enquadra para o contexto é um misto de pesquisa exploratória, na medida em que busca-se criar maior familiaridade com a temática, e pesquisa experimental, pois pretende-se criar condições em ambiente controlado para averiguar determinados casos, com afinalidade de testar o \textit{framework}. A abordagem da pesquisa será quantitativa e qualitativa. Heverá análise dos resultados de forma objetiva, como por exemplo, para cobertura de teste unitário do software alvo. Além disso, será analisado o uso das boas práticas de programação no código-fonte do \textit{framework} o que caracteriza uma abordagem qualitativa.
  \par
  \indent Devido à necessidade da constante participação planejada dos pesquisadores cabe ressaltar o uso da modalidade de pesquisa-ação. Isso conferirá ciclos de coleta e análise de dados e desenvolvimento para alteração do objeto de estudo, conforme as análises do ciclo anterior.

 \subsection{Metodologia de Desenvolvimento do Software} \label{Metodologia de Desenvolvimento do Software}
  No que se refere ao desenvolvimento do software pretende-se utilizar uma adaptação do Scrum, com \textit{Sprints} de quinze dias e algumas de suas práticas, a saber: estimativas relativas, \textit{timebox}, \textit{backlog}, definição de pronto e quadro de tarefas. Procurar-se-á prover o desenvolvimento também com algumas práticas do XP, como \textit{Planning Poker}, Padronização do Código, Integração Contínua e Programação em Pares.
 
 \subsection{Fluxo de Atividades}
 	
 	Utilizando as metodologias definidas para a execução do Trabalho de Conclusão de Curso 01 e 02, foi estabelecido um fluxo de atividades que possibilitará o alcance dos objetivos do projeto.
 	
 \par
 \indent Inicialmente, foi realizado estudos sobre a temática escolhida. Em seguida, foi determinado o escopo do projeto e a proposta inicial resumindo os objetivos, justificativa, cronograma preliminar, entre outros aspectos introdutórios que guiaram as atividades seguintes. O estudo e definição  da metodologia utilizada, bem como o do suporte tecnológico ocorreram em paralelo com a pesquisa e documentação do referencial teórico base para o desenvolvimento do \textit{framework} proposto. Após essas definições, foi realizado a primeira versão detalhada da proposta do software a ser implementado. Definiu-se também uma prova de conceito, de forma a auxiliar o refinamento da proposta. Por fim, houve a finalização da monografia com as atualizações que se fizeram necessárias. A finalização do Trabalho Conclusão de Curso 01 ocorrerá após a primeira apresentação para a banca avaliadora.
 \par
 \indent O Trabalho de Conclusão de Curso 02 iniciará com as correções e refinamentos solicitados na apresentação. Em seguida,  executar-se-á o desenvolvimento do \textit{framework} proposto, utilizando pesquisa-ação e adaptação do Scrum como detalhados nas seções \ref{Metodologia de Pesquisa} e  \ref{Metodologia de Desenvolvimento do Software}, respectivamente. Será efetuada uma análise dos resultados obtidos, verificando se os objetivos estabelecidos no início do projeto foram satisfatoriamente atendidos. Por fim, será adicionado à monografia um relato do que foi executado junto com a análise de resultados.  A apresentação do Trabalho de Conclusão de Curso 02 representará o fim do projeto.
 	
  \begin{figure}[h]
    \centering
    \includegraphics[width=\textwidth]{figuras/processo.jpg}
    \caption{Fluxo de Atividades do TCC}
    \label{fig:processo}
  \end{figure}
 
 \subsection{Cronograma}
 
	Os cronogramas a seguir evidenciam o plano de execução, em relação ao tempo, das atividades referentes ao Trabalho de Conclusão de Curso 01 e 02. Por ser uma visão preliminar, os cronogramas podem ser alterados caso haja necessidade. 
	  
  \begin{figure}[h]
    \centering
    \includegraphics[width=\textwidth]{figuras/cronograma_tcc1.png}
    \caption{Cronograma para o TCC1}
    \label{fig:cronograma_tcc1}
  \end{figure}
  
  \begin{figure}[h]
    \centering
    \includegraphics[width=\textwidth]{figuras/cronograma_tcc2.png}
    \caption{Cronograma para o TCC2}
    \label{fig:cronograma_tcc2}
  \end{figure}