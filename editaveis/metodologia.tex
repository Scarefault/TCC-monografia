\chapter[Metodologia]{Metodologia}

Esse capítulo aborda diversos tipos de metodologias de pesquisa, incluindo a que será utilizada durante o desenvolvimento da pesquisa deste trabalho. 

  \section{Classificações de Metodologias de Pesquisa}
  
Pesquisar pode ser entendido como uma realização concreta de uma investigação planejada, desenvolvida e redigida de acordo com as normas da metodologia consagradas pela ciência, ou, de forma mais simples, como uma atividade para solucionar problemas (livrodemetodologia de pesquisa e RUIZ, 1991). É possível classificar os tipos de pesquisa quanto a área da ciência, natureza, objetivos, procedimentos, objeto de estudo e forma de abordagem. (wiliam)  

Utilizando objetivo geral como critério de classificação, pode-se categorizar pesquisas em três grupos: exploratória, descritiva e explicativa. A pesquisa exploratória busca obter maior conhecimento sobre um assunto específico ainda pouco conhecido tornando-o mais explícito e orientando a formulação de hipóteses (GIl). Geralmente é realizada em forma de estudo de caso ou pesquisa bibliográfica (wiliam) e, para sua execução, costuma-se fazer uso de entrevistas com pessoas experientes no assunto, levantamento bibliográfico e análise de exemplos relacionados com a temática (Gil). 

 \par
  \indent A pesquisa descritiva propõe ao inverstigador descrever as características de uma população ou fenômeno, ou determinar a existência de relações de dependência entre variáveis (Gil).  Em geral, é realizada com auxílio de coleta de dados por meio de questionários e observações sistemáticas (Tafner Silva). Para uma correta interpretação da pesquisa, é importante que o investigador observe, registre, analise e classifique os dados de forma que não haja interferências com os fatos observados (will). 

 \par
  \indent O terceiro grupo refere-se a pesquisa explicativa. Essa tem como objetivo explanar sobre os fatores que determinam ou contribuem com  a ocorrência de um fenômeno específico. Essa categoria de pesquisa visa compreender os motivos de um fenômeno, enquanto a exploratória e descritiva motivam-se a entender o fenômeno em si e detalha-lo (Gil). Comumente é realizada por meio de pesquisa \textit{ ex-post-facto} e pesquisa experimental (Tafner e Silva). 

% Sobre os Procedimentos

% Planejamento da metodologia 
%% Metodologia da Pesquisa Aplicada
%%% Processo bizagi
%% Metodologia do Desenvolvimento Aplicada
%%% Processo bizagi

% Cronogramas Macros

% Prova de Conceito

% Resumo do Capítulo

  \section{Metodologia}
  Tendo em vista o objetivo geral do trabalho, o desenvolvimento de um \textit{framework} para a geração de testes unitários, observa-se que o modelo de pesquisa que mais se enquadra para o contexto é um misto de pesquisa exploratória, na medida em que busca-se criar maior familiaridade com a temática, e pesquisa experimental, pois pretende-se criar condições em ambiente controlado para averiguar determinados casos, com afinalidade de testar o \textit{framework}.
  \par
  \indent Devido à necessidade da constante participação planejada dos pesquisadores cabe ressaltar o uso da modalidade de pesquisa-ação. Isso conferirá ciclos de coleta e análise de dados e desenvolvimento para alteração do objeto de estudo, conforme as análises do ciclo anterior.
  \par
  \indent No que se refere ao desenvolvimento do software pretende-se utilizar uma adaptação do Scrum, com \textit{Sprints} de quinze dias e algumas de suas práticas, a saber: estimativas relativas, \textit{timebox}, \textit{backlog}, definição de pronto e quadro de tarefas. Procurar-se-á prover o desenvolvimento também com algumas práticas do XP, como \textit{Planning Poker}, Padronização do Código, Integração Contínua e Programação em Pares.
  \newpage
  \begin{figure}[h]
    \centering
    \includegraphics[width=\textwidth, natwidth=681, natheight=439]{phases_of_research.png}
    \caption{Fases da Pesquisa}
    \label{fig:phases_of_research}
  \end{figure}
  \indent Considerando-se o escopo do trabalho e a metodologia que se pretende utilizar, as etapas, constantes na Figura \ref{fig:phases_of_research} serão guias do processo de pesquisa.
