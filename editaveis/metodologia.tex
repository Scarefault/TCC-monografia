\chapter[Metodologia]{Metodologia}

%Resumo

% Sobre as metodologias
%Procedimentos

% Planejamento da metodologia 
% metodologia da Pesquisa e porquê
% Metodologia do Desenvolvimento e porquê

%%

%Conclusão do Capítulo

\begin{methodology}
  \section{Metodologia}
  Tendo em vista o objetivo geral do trabalho, o desenvolvimento de um \textit{framework} para a geração de testes unitários, observa-se que o modelo de pesquisa que mais se enquadra para o contexto é um misto de pesquisa exploratória, na medida em que busca-se criar maior familiaridade com a temática, e pesquisa experimental, pois pretende-se criar condições em ambiente controlado para averiguar determinados casos, com afinalidade de testar o \textit{framework}.
  \par
  \indent Devido à necessidade da constante participação planejada dos pesquisadores cabe ressaltar o uso da modalidade de pesquisa-ação. Isso conferirá ciclos de coleta e análise de dados e desenvolvimento para alteração do objeto de estudo, conforme as análises do ciclo anterior.
  \par
  \indent No que se refere ao desenvolvimento do software pretende-se utilizar uma adaptação do Scrum, com \textit{Sprints} de quinze dias e algumas de suas práticas, a saber: estimativas relativas, \textit{timebox}, \textit{backlog}, definição de pronto e quadro de tarefas. Procurar-se-á prover o desenvolvimento também com algumas práticas do XP, como \textit{Planning Poker}, Padronização do Código, Integração Contínua e Programação em Pares.
  \newpage
  \begin{figure}[h]
    \centering
    \includegraphics[width=\textwidth, natwidth=681, natheight=439]{phases_of_research.png}
    \caption{Fases da Pesquisa}
    \label{fig:phases_of_research}
  \end{figure}
  \indent Considerando-se o escopo do trabalho e a metodologia que se pretende utilizar, as etapas, constantes na Figura \ref{fig:phases_of_research} serão guias do processo de pesquisa.
\end{methodology}