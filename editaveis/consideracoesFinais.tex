\chapter[Considerações Finais]{Considerações Finais}
O trabalho aponta a necessidade e as exigências do mercado de software
sobre a qualidade dos produtos e serviços na área. Demonstra os problemas
e alguns dos motivos da falta da produção de testes unitários, uma técnica
fundamental para evitar \textit{bugs} e dificuldades no desenvolvimento
de software.

O \framework desenvolvido nesse trabalho foi proposto com o objetivo de
facilitar a geração de testes unitários para diversas linguagens de programação.
A questão que buscou-se responder foi: \textit{há como propiciar ao desenvolvedor
um suporte que forneça a geração de testes unitários, de forma
semiautomatizada, e que permita adaptações, com o intuito de ajustar-se ao
código-fonte?} Para resolver esse questionamento, foi planeado um 
sistema extensível capaz de gerar testes unitários com uma intervenção breve
do desenvolvedor. No decorrer do projeto fez-se uso do planejamento inicial
para cumprir com as tarefas propostas, como consta o Capítulo 3. No decorrer
do projeto algumas dificuldades surgiram, em especial a respeito da identificação
da linguagem alvo por meio do \flexcpp e do \textsf{Bisonc++}. Isso teve impactos
no planejamento inicial, mas que foram contornados. O processo de desenvolvimento
foi alinhado com análises breves do que deveria ser alterado na produção do
\framework para alcançar um resultado satisfatório. Essa análise se dava nas
conversas entre os dois autores, que identificaram pontos de melhoria no decorrer
do trabalho. A finalização do esforços no desenvolvimento do \framework culminaram
nos resultados expostos no Capítulo 5. Por meio da análise dos resultados
atingidos para cada objetivo específico, é possível concluir que o projeto, de
modo geral, obteve sucesso.

%% Sobre o Curso
O desenvolvimento desse projeto permitiu aplicar diversos conceitos e conhecimentos
adquiridos durante o curso de Engenharia de Software. Entre eles, destacam-se a
Arquitetura de Software, Requisitos de Software, Gerência e Configuração de Software
e Testes de Software. No que diz respeito aos conceitos relativos à Arquitetura de
Software, estão presentes o uso de padrões de projeto (\textit{design patterns}), e
conceitos de orientação a objetos, como herança e polimorfismo. O entendimento desses
conceitos permitiu o avanço e o desenvolvimento da arquitetura do \Scarefault, a ponto de
produzi-lo para que tivesse extensibilidade suficiente para que pudesse ser classificado
como \textit{framework}.

Os conceitos aprendidos sobre  requisitos de software tiveram papel importante no início
do projeto, na medida em que por meio deles foram estabelecidas metas para os
ciclos de desenvolvimento. A Gerência e Configuração de Software permitiu a
automatização de diversos elementos que exigiam tempo durante a produção,
como a compilação do código e o levantamento do ambiente de desenvolvimento.


%%Os Próximos passos
Os autores observam que ainda há melhorias a serem feitas no \Scarefault. Algumas
sugestões para trabalhos futuros são:

\begin{description}
\item[Desenvolver um interpretador próprio:] a seleção do \flexcpp e do \bisoncpp para
auxílio desse trabalho se tornou correta. Isso porque o foco do trabalho foi o
desenvolvimento do \framework voltado para geração de testes. Os autores decidiram
por usar ferramentas que já forneceriam meios de interpretar a linguagem alvo para,
assim, não terem que produzir esse interpretador. No entanto, o uso dessas
dependências prova-se inviável no futuro, pois engessa o \framework no que diz
respeito a interpretação de linguagens. Exige um esforço significativo, o qual deve ser evitada
para que o \scarefault possa evoluir mais. Assim, é visto com bons olhos o
desenvolvimento de um interpretador próprio que seja flexível e extensível, como
parte do \textit{framework}.
\item[Desenvolver \textit{hotspots} para geração de testes inválidos:] atualmente,
o \scarefault cobre testes com valores válidos. É interessante que haja também a
possibilidade de gerar testes para valores inválidos.
\item[Produzir outras estratégias para geração de testes:] o \scarefault gera
testes de caixa preta. Para isso, utiliza-se da estratégia de valores limites. A
adição da possiblidade do uso de outras técnicas de derivação de testes caixa preta
seria bem vinda. Além disso, seria bom que houvesse a possibilidade da geração
de testes de caixa branca.
\end{description}
