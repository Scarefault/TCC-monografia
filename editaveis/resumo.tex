\begin{resumo}
A produção de testes tem se tornado uma atividade importante no processo de desenvolvimento de software. Não apenas o mercado vem exigindo maior qualidade, mas o usuário comum vem demandando redução de falhas nos serviços e produtos de software. O presente trabalho busca a implementação de um \textit{framework} capaz de dar suporte ao desenvolvedor na tarefa de produzir testes unitários. Nesse intuito, um estudo sobre o estado da arte e os conceitos utilizados será efetuado, conferindo insumo para a produção de uma prova de conceito. Em seguida um estudo sobre a melhor arquitetura e técnicas para a geração de testes unitários será conduzido, finalizando com uma investigação sobre a melhor metodologia de pesquisa e desenvolvimento para por em prática os estudos dessa primeira etapa

\vspace{\onelineskip}
    
\noindent
\textbf{Palavras-chaves}: Testes Unitários. \textit{Framework}. Semi Automatização. Verificação e Validação.
\end{resumo}
