\chapter[Referencial Teórico]{Referencial Teórico}
Neste capítulo são apresentadas as bases teóricas para o desenvolvimento do \textit{framework}. O capítulo está organizado em seções. Na seção 2.1 faz-se referência ao contexto de Qualidade de Software. Na seção 2.2 aborda-se uma temática mais específica da qualidade de software, os testes. Traz a motivação para fazê-los, bem como uma descrição sobre os níveis de testes e as vantagens e estratégias de automatizá-los. Na seção 2.3 dicuti-se sobre \textit{frameworks}, sua definição, motivação, vantagens e desvantagens e classificação. Na seção 2.4 explana-se sobre a geração de testes e técnicas de implementá-la.

\section{Qualidade de Software}
Software é uma atividade que requer um processo com o envolvimento de diversas atividades e diferentes pessoas. Esse contexto permite a inserção de defeitos no produto final \cite{trodo_2009}. Além disso, há o risco de produzir-se algo que não foi solicitado, devido ao mal entendimento dos requisitos\textsuperscript{Falta referência}. Esse conjunto de fatores fez surgir maior preocupação em relação a qualidade de software e, inclusive nesse contexto, é que há o advento da Engenharia de Software, com o objetivo de produzir softwares com qualidade \cite{bueno_e_campelo_2013}.

%  Testes
%    Teste Aceitação
%    Teste Sistema
%    Teste Integração
%    Teste Unitário
%    Automatização
% Framework
% Geração de Testes
% Resumo do Capítulo