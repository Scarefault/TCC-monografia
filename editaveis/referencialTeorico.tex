\chapter[Referencial Teórico]{Referencial Teórico}
Neste capítulo são apresentadas as bases teóricas para o desenvolvimento do \textit{framework}. O capítulo está organizado em seções. Na seção 2.1 faz-se referência ao contexto de Qualidade de Software. Na seção 2.2 Aborda-se uma temática mais específica da qualidade de software, os testes. Traz a motivação para fazê-los, bem como uma descrição sobre os níveis de testes e as vantagens e estratégias de automatizá-los. Na seção 2.3 Dicuti-se sobre \textit{frameworks}, sua definição, motivação, vantagens e desvantagens e classificação. Na seção 2.4 explana-se sobre a geração de testes e técnicas de implementá-la.

% Qualidade de Software
%  Testes
%    Teste Aceitação
%    Teste Sistema
%    Teste Integração
%    Teste Unitário
%    Automatização
% Framework
% Geração de Testes
% Resumo do Capítulo