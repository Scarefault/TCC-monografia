\chapter[Referencial Teórico]{Referencial Teórico}
Neste capítulo são apresentadas as bases teóricas para o desenvolvimento do \textit{framework}. O capítulo está organizado em seções. Na seção 2.1 faz-se referência ao contexto de Verificação e Validação de Software. Na seção 2.2 aborda-se uma temática mais específica da qualidade de software, os testes. Traz a motivação para fazê-los, bem como uma descrição sobre os níveis de testes e as vantagens e estratégias de automatizá-los. Na seção 2.3 dicuti-se sobre \textit{frameworks}, sua definição, motivação, vantagens e desvantagens e classificação. Na seção 2.4 explana-se sobre a geração de testes e técnicas de implementá-la.

\section{Verificação e Validação de Software}
Software é uma atividade que requer um processo com o envolvimento de diversas atividades e diferentes pessoas. Essa característica permite a inserção de defeitos no produto final \cite{trodo_2009}. Além disso, há o risco de produzir-se algo que não foi solicitado, devido ao mal entendimento dos requisitos\textsuperscript{Falta referência}. Esse conjunto de fatores fez surgir maior preocupação em relação a qualidade de software e, inclusive nesse contexto, é que há o advento da Engenharia de Software, com o objetivo de produzir softwares com qualidade \cite{bueno_e_campelo_2013}.
\par
\indent Tendo em vista o cuidado com o nível dos softwares surge a necessidade de conceituar qualidade no que tange esse assunto. Essa definição é díficil, pois qualidade é um conceito abstrato. Em essência, a qualidade esta ligada a possibilidade de medir determinado atributo e comparar o resultado com padrões já conhecidos \cite{bueno_e_campelo_2013}. No âmbito de software há também o conceito de métricas. Elas são utilizadas para dar visibilidade de determinadas características do produto, de forma a demonstrar o tamanho, esforço e complexidade do software em construção, dentre outros atributos \cite{abreu_2011}.
\par
\indent Tendo isso em vista, observa-se que, no que se refere ao software, tem-se como caracterizar a qualidade em dois tipos, na medida em que há características mensuráveis tanto no projeto quanto no produto. São eles: qualidade de projeto e qualidade de conformidade. O primeiro diz respeito aos requisitos, especificações e arquitetura, enquanto que o segundo foca na implementação e sua conformidade com o que foi especificado \cite{bueno_e_campelo_2013}.
\par
\indent A qualidade de software está ligada a atividade de verificação e validação de software\textsuperscript{Falta referência}. A qualidade de software é uma atividade que pertence ao ambiente de gerência, enquanto que a verificação e validação de software é uma prática mais técnica e enquadra-se no desenvolvimento do produto \cite{bueno_e_campelo_2013}. Segundo \citeonline{sommerville_2007}, verificação e validação de software é o processo, que certifica que o produto em desenvolvimento atende as especificações esperadas pelo cliente. Essa atividade deve permear todo o processo de produção do software, de forma a garantir que o produto respeita o especificado desde o início. essa prática evita que a identificação de defeitos seja percebido apenas ao final do desenvolvimento, o que aumentaria os custos do projeto.

%  Testes
%    Teste Aceitação
%    Teste Sistema
%    Teste Integração
%    Teste Unitário
%    Automatização
% Framework
% Geração de Testes
% Resumo do Capítulo