\chapter[Introdução]{Introdução}

Neste capítulo, serão descritos o contexto no qual se insere o trabalho, a questão de pesquisa a ser explorada, a justificativa para propor um estudo no tema, e os objetivos.

 \section{Contextualização}
  \indent A produção de software é uma atividade que, devido à sua natureza abstrata, vem acompanhada da eventual inserção de defeitos no código \cite{trodo2009}. A partir da década de 1990, os usuários de software passaram a exigir maior atenção por parte das empresas desenvolvedoras em relação à redução de falhas de seus produtos e serviços \cite{sommerville2007}. Além disso, a demanda por software no mercado está crescendo \cite{philipson2004} e, por conseguinte, a exigência de maior qualidade no resultado final do processo de desenvolvimento aumenta \cite{barbosaEtAl2009}.
  \par
  \indent Nesse contexto, a identificação de defeitos que casualmente estejam presentes no código é uma preocupação incessante no processo de produção de software. Para essa finalidade, surge a prática de testes do produto. Em essência, essa etapa visa à execução do software alvo com dados de teste, de forma a verificar se os resultados observados correspondem à expectativa. Isso permite ao desenvolvedor demonstrar aos seus clientes que o software está de acordo com as suas especificações, além de viabilizar ao programador a busca e descoberta de defeitos no software \cite{sommerville2007}.
  \par
  \indent A prática de testes de software tornou-se parte importante do processo de desenvolvimento \cite{barbosaEtAl2009}. Diversas categorias de testes, com diferentes intenções de análise sobre o produto, foram surgindo, a saber: testes de unidade, testes de integração e testes de aceitação. Testes de integração são aqueles que permitem a observação de que os componentes que formam o software funcionam corretamente em conjunto. Testes de aceitação garantem que as interfaces do sistema estão de acordo com o especificado, bem como o comportamento esperado por elas. Os testes de unidade têm como proposta a garantia  de que uma determinada parte (unidade) do código esteja respondendo como o esperado \cite{sommerville2007}.
  \par
  \indent Outra questão que surge na produção de software, e que é um fator importante a ser considerado, é o custo que \textit{bugs} geram no desenvolvimento e na fase de manutenção. Estimativas apontam que os gastos com a correção de \textit{bugs} chegam a mais de 59,5 bilhões de dólares anuais nos Estados Unidos \cite{jantti2008}. No Brasil, gasta-se 70\% do tempo de desenvolvimento corringindo-se erros \cite{janones2010}. Esse cenário aumenta o preço do produto final.
  \par
  \indent A produção de testes afeta positivamente o desenvolvimento de software não apenas no nível estratégico. Segundo \citeonline{burkeCoyner2003}, há diversas razões para que se escreva testes unitários, dentre elas:
  \begin{itemize}
    \item Testes reduzem defeitos em funcionalidades novas e já existentes.
    \item Testes auxiliam na documentação do código.
    \item Testes permitem refatoração com maior qualidade.
    \item Testes reduzem o receio de alterar o código.
    \item Testes defendem o código contra alterações indesejáveis de outros programadores.
  \end{itemize}

 \section{Questão de Pesquisa}
O intuito deste trabalho de conclusão de curso é auxiliar os desenvolvedores de software em relação à seguinte questão: há como propiciar ao desenvolvedor um suporte que forneça a geração de testes unitários, de forma semiautomatizada, e que permita adaptações, com o intuito de ajustar-se ao código-fonte?

 \section{Justificativa}
  Testes são cruciais no desenvolvimento de software, como evidenciado na seção anterior. Contudo, de maneira geral, desenvolvedores não escrevem testes para os seus programas \cite{burkeCoyner2003}. As razões são variadas, argumentando que não sabem escrever testes ou que não têm tempo para fazê-los \cite{burkeCoyner2003}. Tendo em vista este cenário, onde o mercado de software torna-se cada vez mais exigente com relação à qualidade dos produtos e serviços, bem como os altos custos relacionados à falta de empenho e previdência sobre a qualidade dos sistemas produzidos, é contraditório observar que a prática de fazer testes não seja comum, ou mesmo prioritária por parte dos programadores.
  \par
  \indent No artigo de \citeonline{burkeCoyner2003}, intitulado \textit{Top 12 Reasons to Write Unit Tests}, os autores revelam que algumas das desculpas mais frequentes que já ouviram em suas carreiras para os programadores não fazerem testes de unidade são:
  \begin{itemize}
    \item \textit{"Eu não sei escrever testes."}
    \item \textit{"Escrever testes é muito difícil."}
    \item \textit{"Não tenho tempo suficiente para fazer testes."}
    \item \textit{"Testes não são o meu trabalho."}
  \end{itemize}
  \par
  \indent Essa lista evidência um fato já conhecido no desenvolvimento de software: a produção de testes é uma atividade onerosa \cite{barbosaEtAl2009}. Esse cenário é intricado, pois há evidências dos benefícios da produção de testes e, no entanto, há um distanciamento dos desenvolvedores em relação aos testes unitários. Considerando esse panorama, a elaboração de um suporte capaz de apoiar o programador na tarefa de gerar os testes unitários, reduzindo o esforço e os custos associados, será de uma valia considerável, tanto no âmbito técnico quanto estratégico no processo de desenvolvimento de software.
 
  \section{Objetivos}
  Esse trabalho visa alcançar os objetivos, Geral e Específicos, apresentados a seguir.
    \subsection{Objetivo Geral}
    Desenvolver um \textit{framework} capaz de dar suporte aos engenheiros de software na geração - semiautomatizada - de testes unitários.
    
    \subsection{Objetivos Específicos}
    Os seguintes itens são considerados importantes, devido à sua relevância para o entendimento de teste de software e aplicação dos conhecimentos sobre engenharia de software e, portanto, fazem parte dos objetivos específicos desse trabalho. São eles:

    \begin{enumerate}
      \item Aprofundar o conhecimento na área de Testes de Software.
      
      \item Investigar abordagens associadas ao tema foco desse trabalho de conclusão de curso, via revisão bibliográfica e provas de conceito, no intuito de compilar soluções candidatas ao desenvolvimento do \textit{framework} de geração de testes de unidade.
      
      \item Aplicar métodos, técnicas e boas práticas de Engenharia de Software no processo de desenvolvimento do \textit{framework}.
      
      \item Gerar testes unitários por meio do \textit{framework} que cubram métodos simples, como criar, recuperar, atualizar e apagar.

      \item Coletar as primeiras impressões dos testes gerados pelo \textit{framework} e documentá-las, afim de facilitar a evolução no futuro.

    \end{enumerate}
    
\section{Organização do Documento}
	Este documento está dividido da seguinte forma:
	
	\begin{description}
		\item[Capítulo 2 - Referencial Teórico:] apresenta conceitos, modelos e abordagens associados ao tema foco desse trabalho;
		\item[Capítulo 3 - Suporte Tecnológico:] caracteriza as recursos tecnólogicos utilizados durante a elaboração desse trabalho;
		\item[Capítulo 4 - Proposta:] descreve a proposta de desenvolvimento de um \textit{framework} para geração de testes unitários;
		\item[Capítulo 5 - Metodologia:] especifica a metodologia utilizada para pesquisa e desenvolvimento deste trabalho;
		\item[Capítulo 6 - Resultados Obtidos:] apresenta os resultados obtidos até o momento na implementação do framework proposto.
	\end{description}