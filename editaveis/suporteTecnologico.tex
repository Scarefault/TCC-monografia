\chapter[Suporte Tecnológico]{Suporte Tecnológico}

Este capítulo apresenta um compilado das principais ferramentas que serão utilizadas no desenvolvimento do \textit{framework} para geração de teste unitário. Com a finalidade de melhor organizar o capítulo, os recursos tecnológicos foram divididos em duas categorias: Geração de Teste Unitário e Engenharia de Software. A primeira refere-se aos instrumentos que serão utilizados na construção do \textit{framework}, enquanto a segunda categoria visa auxiliar o gerenciamento e desenvolvimento do software e documentação relacionada.

\section{Geração de Teste Unitário}

Os recursos tecnológicos identificados como capazes de auxiliar no desenvolvimento do \textit{framework} estão listada abaixo:

\subsection{Flex}
Flex é uma ferramenta capaz de reconhecer padrões lexicais no texto. Criado por Jef Poskanzer na linguagem Ratfor e, anos mais tarde, traduzido para linguagem C por Vern Paxson em meados de 1987, flex é software livre distribuído pela \textit{Free Software Foundation} \cite{flex}. 

\subsection{Bison}
Bison é um gerador de interpretadores de linguagens. Essa ferramenta está disponível sob a licença \textit{General Public License} (GPL) e possui compatibilidade com o Flex \cite{bison}. 

\section{Engenharia de Software}

As ferramentas a seguir serão utilizadas para auxiliar no gerenciamento e implementação, bem como a documentação associada. Buscou-se aplicar os conhecimentos adquiros ao decorrer do curso de engenharia de software para apoiar o processo, tanto em tecnologia, quanto em práticas de desenvolvimento de software.

\subsection{Git}
Git é um sistema de versão distribuído de código aberto e gratuito. Criado em 2005, Git foi desenvolvido visando manter velociadade e eficiência para pequenos e grandes projetos \cite{git}. 

\subsection{Bitbucket}
Bitbucket é um serviço de hospedagem de projetos de software que utiliza Git ou Mercurial como controle de versão, dependendo da escolha do usuário. Bitbucket fornece, entre suas funcionalidades, o gerenciamento de \textit{issues}, uso de \textit{wiki}, controle de permissão de \textit{branchs}, revisão de código e utilização gratuita de repositórios privados. A empresa Atlassian é responsável pela manutenção e disponibilização desse sistema \cite{bitbucket}.

\subsection{Jenkins}
Jenkins é um sistema de integração contínua capaz de gerar \textit{builds} periódicos e automáticos, facilitando a integração de alterações em um projeto. Multi-plataforma e software livre, Jenkins permite integração com diversos \textit{plugins} de forma a facilitar sua customização para o uso \cite{jenkins}.

\subsection{Atom}
Atom é um editor de texto e código fonte disponível para Linux, Windows e Mac OS X. É um ambiente de desenvolvimento livre e de código aberto disponibilizado pela empresa GitHub \cite{atom}. 

\subsection{Ubuntu}
Distribuição linux, livre e de código aberto. Patrocinado pela Canonical Ltda., ubuntu utiliza kernel Linux baseado em Debian e é licenciado pela \textit{General License Public} (GPL) \cite{ubuntu}. A versão utilizada foi a 14.04.  

\subsection{LaTeX}
Criado inicialmente em 1985 por Leslie Lamport, o LaTeX é um sistema de preparação de documentos para composição tipográfica que tem como principal objetivo reduzir o esforço dos autores na formatação de textos técnicos ou científicos. Baseado na linguagem TeX, criada por Donald E. Knuth, LaTeX está disponível como software livre e é mantido pela \textit{LaTeX3 Project} \cite{latex}.

\subsection{Trello} 
Serviço online que permite o gerenciamento de projetos e tarefas por meio de quadros e listas. Trello possui uma interface simples que permite anexo de arquivo, compartilhamento de conteúdo por equipe, adicionar responsável em tarefas,  entre outras funcionalidades. Embora haja recursos pagos, não há necessidade de pagamento para utilizar essa ferramenta \cite{trello}.

\subsection{TexMaker}
Editor de texto multi-plataforma para LaTex. Software livre licenciado pela General Public License (GPL), o TexMaker inclui suporte \textit{unicode}, verificação ortográfica, função de auto-completar, visualizador de arquivo em PDF, além do editor para escrita de arquivos LaTex \cite{texmaker}. Foi utilizada a versão 4.1.

\subsection{Zotero}
Zotero é um software livre (GPLv3), criado e mantido pela \textit{George Mason University}, que tem por finalidade a gerência de bibliografias e materiais relacionados a pesquisa. Entre suas principais funcionalidades estão a geração de relatório de referências e a exportação de bibliografias já formatadas \cite{zotero}. A versão utilizada foi a 4.0.28.

\subsection{Bizagi Modeler}
Ferramenta gratuita para modelagem de processos que respeita a especificação BPMN (\textit{Business Process Modeling Notation}). Bizagi fornece suporte de ajuda para usuário com guias e tutoriais, além de permitir a edição compartilhada de processos entre suas funcionalidades \cite{bizagi}. Foi utilizada a versão 2.9.

\section{Resumo do Capítulo}
Este capítulo explanou sobre o suporte tecnológico necessário para a elaboração deste Trabalho de Conclusão de Curso. Foram abordadas tanto as ferramentas que auxiliam diretamenta na implementação do \textit{framework}, quanto as que apoiam o processo de gestão e desenvolvimento de software.
