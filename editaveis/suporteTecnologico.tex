\chapter[Suporte Tecnológico]{Suporte Tecnológico}
Este capítulo apresenta um compilado das principais ferramentas que serão utilizadas no desenvolvimento do \textit{framework} para geração de testes unitários. Com a finalidade de melhor organizar o capítulo, os recursos tecnológicos foram divididos em categorias: Geração de Teste Unitário e Engenharia de Software. A primeira refere-se aos suportes que serão utilizados na construção do \textit{framework}, enquanto a segunda categoria visa auxiliar o gerenciamento e desenvolvimento do software e documentação relacionada.

\section{Geração de Teste Unitário} \label{suporteGeracao}
Os recursos tecnológicos identificados como capazes de auxiliar no desenvolvimento do \textit{framework} estão apresentados a seguir:

\subsection{Flexc++}
Flexc++ é um analisador léxico escrito por Frank B. Brokken, gerente de segurança e conferencista na \textit{University of Groningen}, Holanda\cite{flexcpp2015}. O desenvolvimento desse analisador começou em 2008 e é mantido até hoje. Atualmente está na versão 2.03.03, é mantido por: Frank B. Brokken, Jean-Paul van Oosten e Richard Berendsen \cite{flexcpp2015}. No entanto, a versão utilizada será a versão 1.05. Basicamente, a diferença é que Flexc++ tem por intuito gerar código em C++ para que possa ser usado juntamente com programas escritos em C++. Enquanto isso, Flex gera código escrito em C e Flex++ dificilmente dar suporte a todo o pontencial que C++ oferece hoje.

\subsection{Bisonc++}
Bisonc++ é um gerador de parsers capaz de converter uma gramática de contexto livre em uma classe \textit{parser} em C++ \cite{bisoncpp2015}. Bisonc++ foi escrito e é mantido por Frank B. Brokken, da \textit{University of Groningen}, Holanda. Bisonc++, diferentemente do Bison, fornece uma classe em C++ após ser compilado para que possa ser usada. Bison fornece um código em C \cite{bisoncpp2015}.
\par
\indent Ambas as ferramentas mencionadas, Flexc++ e Bisonc++, são muito utilizadas no contexto de compiladores \cite{aaby2004}. Entretanto, por darem suporte ao reconhecimento léxico e à interpretação de linguagens, são candidatos a fornecerem auxílio no desenvolvimento do \textit{framework}. Para a geração de testes é necessário o reconhecimento de palavras-chave capazes de revelar informações sobre os métodos a serem testados. Também darão suporte na interpretação dos arquivos de código fonte, bem como no desenvolvimento de um \textit{parser}, cujo produtos finais serão os testes gerados.

\section{Engenharia de Software}
As ferramentas a seguir serão utilizadas para auxiliar no gerenciamento e na implementação, bem como na documentação associada. Buscou-se aplicar os conhecimentos adquiridos no decorrer do curso de engenharia de software para apoiar o processo, tanto em tecnologia, quanto em práticas de desenvolvimento de software.

\subsection{Gerência de Projetos}
No que tange à área de gerência de projetos, as seguintes ferramentas serão utilizadas:

\subsubsection{Trello} 
Serviço \textit{online} que permite o gerenciamento de projetos e tarefas por meio de quadros e listas. Trello possui uma interface simples que possibilita anexar arquivos, compartilhamento de conteúdo por equipe, adicionar responsável em tarefas,  entre outras funcionalidades. Embora haja recursos pagos, não há necessidade de pagamento para utilizar essa ferramenta \cite{trello2015}.
\par
\indent O Trello irá centralizar as informações gerenciais do projeto. As tarefas a serem executadas, descrição e prazos estarão disponíveis nessa ferramenta.

\subsubsection{Bizagi Modeler}
Ferramenta gratuita para modelagem de processos que respeita a especificação BPMN (\textit{Business Process Modeling Notation}). Bizagi fornece suporte de ajuda para usuário com guias e tutoriais, além de permitir a edição compartilhada de processos entre suas funcionalidades \cite{bizagi2015}. Foi utilizada a versão 2.9.
\par
\indent O Bizagi Modeler permitirá, em um momento inicial do projeto, a modelagem de um processo que irá guiar o trabalho de conclusão de curso. O modelo desse processo permite a visualização e o entendimento do processo seguido.

\subsection{Desenvolvimento de Software}
Em relação ao desenvolvimento de software, os seguintes suportes foram selecionados:

\subsubsection{Vim}
Vim é um editor de texto configurável construído para permitir a edição de texto de forma eficiente. Essa ferramenta foi criada com base no editor Vi, distribuído com a maioria dos sistemas UNIX \cite{vim}. Licenciado pela \textit{charityware}, Vim é distribuído gratuitamente. 
\par
\indent Esse editor de texto será utilizado para a edição dos arquivos de código fonte.

\subsubsection{Ubuntu}
Distribuição linux, livre e de código aberto. Patrocinado pela Canonical Ltda., Ubuntu utiliza kernel Linux baseado em Debian e é licenciado pela \textit{General License Public} (GPL) \cite{ubuntu2010}. A versão utilizada foi a 14.04.
\par
\indent O Ubuntu será utilizado como sistema operacional padrão para o desenvolvimento do \textit{framework}.

\subsection{Gerência de Configuração de Software e Requisitos de Software}
Para dar suporte ao desenvolvimento, automatizando o máximo possível determinadas tarefas, as seguintes ferramentas foram escolhidas para a gerência de configuração de software:

\subsubsection{Git}
Git é um sistema de versão distribuído de código aberto e gratuito. Criado em 2005, Git foi desenvolvido visando manter velocidade e eficiência para pequenos e grandes projetos \cite{git2015}. 
\par
\indent O Git permitirá o versionamento do código fonte, bem como da monografia, proporcionando controle das versões e possíveis resgates de informações.

\subsubsection{GitLab}
GitLab é um serviço de hospedagem de projetos de software que utiliza Git como controle de versão. GitLab fornece, entre suas funcionalidades, o gerenciamento de \textit{issues}, uso de \textit{wiki}, controle de permissão de \textit{branchs}, revisão de código e utilização gratuita de repositórios privados \cite{gitlab}.
\par
\indent O GitLab será um local central, onde diversas informações sobre o projeto serão encontradas, inclusive o próprio código. É nesse centro de informações do projeto que também estarão relacionados os requisitos de software, e sua rastreabilidade com as histórias de usuário.

\subsubsection{Jenkins}
Jenkins é um sistema de integração contínua capaz de gerar \textit{builds} periódicos e automáticos, facilitando a integração de alterações em um projeto. Multi-plataforma e software livre, Jenkins permite a integração com diversos \textit{plugins} de forma a facilitar sua customização \cite{jenkins2015}.
\par
\indent O Jenkins garantirá que o software em desenvolvimento esteja sempre sendo testado e integrado, proporcionando que essas atividades não sejam feitas tardiamente. Auxilia, assim, os desenvolvedores a perceberem problemas mais cedo e reagirem rápido.

\subsubsection{Vagrant}
Vagrant é uma ferramenta capaz de gerenciar a criação de máquinas virtuais para ambientes de desenvolvimento. Sua utilização proporciona a automação da instalação e configuração de ferramentas, além de facilitar a padronização dos ambientes de desenvolvimento utilizados por uma equipe de software \cite{vagrant2015}. Inicialmente criado por Mitchell Hashimoto em janeiro de 2010, Vagrant é um software aberto mantido pela atualmente principlamente pela empresa HashiCorp \cite{vagrant2015}. 

A utilização do Vagrant permitirá a padronização do ambiente de desenvolvimento utilizado na implementação do \textit{framework} proposto. Também tornará mais simples a replicação desse ambiente, caso seja necessário.

\subsection{Pesquisa}
Para dar suporte ao andamento da pesquisa e produção da monografia as seguintes ferramentas serão usadas:

\subsubsection{LaTeX}
Criado inicialmente em 1985 por Leslie Lamport, o LaTeX é um sistema de preparação de documentos para composição tipográfica que tem como principal objetivo reduzir o esforço dos autores na formatação de textos técnicos ou científicos. Baseado na linguagem TeX, criada por Donald E. Knuth, LaTeX está disponível como software livre e é mantido pela \textit{LaTeX3 Project} \cite{latex2015}.

\subsubsection{TexMaker}
Editor de texto multi-plataforma para LaTex. Software livre licenciado pela General Public License (GPL), o TexMaker inclui suporte \textit{unicode}, verificação ortográfica, função de auto-completar, visualizador de arquivo em PDF, além do editor para escrita de arquivos LaTex \cite{texmaker2014}. Foi utilizada a versão 4.1.

\subsubsection{Zotero}
Zotero é um software livre (GPLv3), criado e mantido pela \textit{George Mason University}, que tem por finalidade a gerência de bibliografias e materiais relacionados à pesquisa. Entre suas principais funcionalidades estão a geração de relatório de referências e a exportação de bibliografias já formatadas \cite{zotero2015}. A versão utilizada foi a 4.0.28.

\section{Resumo do Capítulo}
Este capítulo explanou sobre o suporte tecnológico necessário para a elaboração deste trabalho de conclusão de curso. Foram abordadas tanto as ferramentas que auxiliam diretamente na implementação do \textit{framework}, quanto as que apoiam o processo de gestão e desenvolvimento de software.
