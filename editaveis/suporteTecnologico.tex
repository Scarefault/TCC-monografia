\chapter[Suporte Tecnologico]{Suporte Tecnológico}

%%Fazer resumo
\section{Geradores...}
%%Flex
%%Bison

\section{Engenharia de Software}

%% Resumo

\subsection{Trello} 
\subsection{Jenkies}
\subsection{LaTeX}
Criado inicialmente em 1985 por Leslie Lamport, o LaTeX é um sistema de preparação de documentos para composição tipográfica que tem como principal objetivo reduzir o esforço dos autores na formatação de textos técnicos ou científicos. Baseado na linguagem TeX, criada por Donald E. Knuth, LaTeX está disponível como software livre e é mantido pela \textit{LaTeX3 Project}.

\subsection{TexMaker}
Editor de texto multi-plataforma para LaTex. Software livre licenciado pela General Public License (GPL), o TexMaker inclui suporte \textit{unicode}, verificação ortográfica, função de auto-completar, visualizador de arquivo em PDF, além do editor para escrita de arquivos LaTex. Foi utilizada a versão 4.1.

\subsection{Zotero}
Zotero é um software livre (GPLv3), criado e mantido pela \textit{George Mason University}, que tem por finalidade a gerência de bibliografias e materiais relacionados a pesquisa. Entre suas principais funcionalidades estão a geração de relatório de referências e a exportação de bibliografias já formatadas. A versão utilizada foi a 4.0.28.

\subsection{Bizagi Modeler}
Ferramenta gratuíta para modelagem de processos que respeita a especificação BPMN (\textit{Business Process Modeling Notation}). Bizagi fornece suporte de ajuda para usuário com guias e tutoriais, além de permitir a edição compartilhada de processos entre suas funcionalidades. Foi utilizada a versão 2.9.

\subsection{Git}
\subsection{Bitbucket}
\subsection{Zotero}
\subsection{Ubuntu}